%% main.tex
%% IEEE Computer Society Journal Paper
%% Based on bare_jrnl_compsoc.tex V1.4b by Michael Shell

\documentclass[10pt,journal,compsoc]{IEEEtran}

\ifCLASSOPTIONcompsoc
  \usepackage[nocompress]{cite}
\else
  \usepackage{cite}
\fi

\usepackage{amsmath}
\usepackage{amssymb}
\usepackage{url}

\hyphenation{op-tical net-works semi-conduc-tor}


\begin{document}

\title{Beyond the M$\times$N Problem: A Comparative Architectural Analysis of\\
Model Context Protocol (MCP) and Agent-to-Agent (A2A) Interaction Patterns}

\author{Ömer Furkan Tercan%
%\IEEEcompsocitemizethanks{%
%  \IEEEcompsocthanksitem F. Tercan is an independent researcher.\protect\\
%  E-mail: furkan@tercan.dev
%}%
%\thanks{Manuscript received \today.}
}

% E-mail: omer.tercan@helsinki.fi

%\markboth{Journal of \LaTeX\ Class Files,~Vol.~1, No.~1, 2026}%
%{Tercan: Beyond the M$\times$N Problem: MCP and A2A Interaction Patterns}

\IEEEtitleabstractindextext{%
\begin{abstract}
The transition from conversational AI to agentic systems has created an
``$M \times N$ integration problem,'' where $M$ models require bespoke
connectors for $N$ data sources, leading to fragmented infrastructure.
While recent surveys have cataloged emerging standards like MCP,
Agent-to-Agent (A2A), and Agent Communication Protocol (ACP), there
remains a lack of rigorous comparison regarding their architectural
trade-offs. This paper analyzes MCP against its primary
contenders---Google's A2A protocol and OpenAI's REST-based function
calling---to determine the specific utility of stateful versus stateless
interaction patterns. We utilize a ``Systematization of Knowledge'' (SoK)
approach combined with a constructive evaluation of a standardized
``Research Assistant'' task graph implemented across three protocols.
Analysis reveals that MCP's persistent, stateful connection model
(JSON-RPC) offers superior context management for local-first and
high-fidelity tool use, whereas A2A excels in decentralized, trust-based
task delegation between autonomous entities. We propose a unified ``Agent
Protocol Stack,'' arguing that MCP and A2A are complementary layers---MCP
as the standard for \emph{tool execution} and A2A for \emph{agent
collaboration}---rather than mutually exclusive competitors.
\end{abstract}

\begin{IEEEkeywords}
Model Context Protocol, MCP, Agent-to-Agent, A2A, ACP, agent
interoperability, JSON-RPC, stateful protocols, Systematization of
Knowledge.
\end{IEEEkeywords}}

\maketitle
\IEEEdisplaynontitleabstractindextext
\IEEEpeerreviewmaketitle


\IEEEraisesectionheading{\section{Introduction}\label{sec:introduction}}

\IEEEPARstart{T}{he} rapid proliferation of Large Language Models (LLMs)
has necessitated standardized interfaces for tools and memory. Current
ad-hoc integrations (manual API wiring, framework-specific wrappers like
LangChain) are brittle and unscalable~\cite{ehtesham:survey2025,
anthropic:mcp2024}.

Previous works, such as ``A Survey of Agent Interoperability
Protocols''~\cite{ehtesham:survey2025}, provided a necessary taxonomy of
the landscape. However, developers currently lack a decision framework for
choosing between \emph{tool-centric} protocols (MCP) and
\emph{agent-centric} protocols (A2A) based on architectural constraints
like latency, security, and state handling~\cite{hou:mcpsecurity2025,
ehtesham:survey2025}.

This paper addresses the following research questions:

\begin{description}
  \item[{RQ1 (Architecture)}] How do MCP's core primitives (Tools,
    Resources, Prompts) diverge from the ``Agent Card'' and capability
    negotiation models of A2A and ACP?
  \item[{RQ2 (Security \& State)}] How does MCP's stateful JSON-RPC session
    model impact security boundaries compared to stateless REST-based
    function calling~\cite{hou:mcpsecurity2025}?
  \item[{RQ3 (Convergence)}] Can these protocols coexist? Is there evidence
    for a layered ``TCP/IP moment'' for the internet of agents?
\end{description}


\section{Background \& Related Work}
\label{sec:background}

\begin{itemize}
  \item \textbf{The ``$M \times N$'' Integration Crisis:} Definition of
    the fragmentation problem where $M$ agents need bespoke connectors for
    $N$ tools, stifling innovation~\cite{anthropic:mcp2024,
    ehtesham:survey2025}.
  \item \textbf{Precursors:} Brief coverage of the Language Server
    Protocol (LSP), which inspired MCP's client-host-server
    topology~\cite{anthropic:mcp2024}, and OpenAI Function Calling, which
    established the de facto REST-based standard~\cite{openai:fc2023}.
  \item \textbf{Existing Surveys:} Cite ``A Survey of Agent
    Interoperability Protocols''~\cite{ehtesham:survey2025} as the
    foundational text. This paper distinguishes itself by moving from
    \emph{description} (what exists) to \emph{architectural evaluation}
    (how it behaves under load/attack).
\end{itemize}


\section{Architectural Analysis}
\label{sec:architecture}

\subsection{Taxonomy of Interaction Patterns}

\begin{itemize}
  \item \emph{Direct Tooling (OpenAI/LangChain):} Ephemeral,
    request-response, stateless. The context must be re-injected every
    turn~\cite{openai:fc2023}.
  \item \emph{Context-First (MCP):} Persistent connections via JSON-RPC
    (over Stdio or SSE). Decouples ``passive context'' (Resources) from
    ``active execution'' (Tools)~\cite{anthropic:mcp2024,oribe:mcp2025}.
  \item \emph{Peer-Delegation (A2A/ACP):} High-level task handoff,
    asynchronous event loops, and trust-based capability negotiation
    between autonomous peers~\cite{google:a2a2024,ibm:acp2024}.
\end{itemize}

\subsection{Comparison Matrix}

\begin{itemize}
  \item \textbf{Transport:} MCP's local-first focus (Stdio) vs.\ A2A's
    web-first design (HTTP/SSE).
  \item \textbf{Discovery:} MCP's \texttt{initialize} handshake and
    capability declaration vs.\ A2A's ``Agent Card'' lookup vs.\ OpenAI's
    static schema definition.
  \item \textbf{State Handling:} MCP's server-driven resource updates
    (subscriptions) vs.\ REST's client-driven
    polling~\cite{li:measurement2025}.
\end{itemize}


\section{Security \& Safety Evaluation}
\label{sec:security}

\subsection{Threat Model Divergence}

\begin{itemize}
  \item \textbf{MCP Risks:} ``Cross-Primitive Escalation'' (using a
    read-only Resource to trigger a Tool action) and ``Rug Pulls''
    (malicious servers changing behavior after trust
    establishment)~\cite{gaire:sok2025,anon:secanalysis2025}.
  \item \textbf{Injection Vectors:} Analysis of ``Indirect Prompt
    Injection'' where malicious content in an MCP Resource (e.g., a
    GitHub issue) hijacks the agent's control
    flow~\cite{gaire:sok2025,hou:mcpsecurity2025}.
\end{itemize}

\subsection{The Human-in-the-Loop Problem}

MCP's \texttt{sampling} primitive allows servers to request LLM
completion, creating a bidirectional control flow that complicates
permission boundaries compared to unidirectional REST
APIs~\cite{anthropic:mcp2024,gaire:sok2025}.


\section{Proposed Convergence: The Agent Protocol Stack}
\label{sec:stack}

This section addresses the ``Future Work'' gap by synthesizing the
protocols into a layered model (the ``TCP/IP'' analogy):

\begin{description}
  \item[{Layer 3 --- Collaboration}] \textbf{A2A / ACP} (``The Social
    Layer''). Agents talking to Agents. Delegating high-level goals
    (``Plan a trip'')~\cite{google:a2a2024,ibm:acp2024}.
  \item[{Layer 2 --- Context \& Tools}] \textbf{MCP} (``The Hands \&
    Eyes''). Agents talking to Data/Tools. Executing specific atomic
    actions (``Query database,'' ``Read file'')~\cite{anthropic:mcp2024}.
  \item[{Layer 1 --- Transport}] HTTP / SSE / JSON-RPC.
\end{description}

MCP and A2A are complementary. An A2A agent (the high-level planner)
effectively acts as an \emph{MCP Host} to execute specific sub-tasks
using \emph{MCP Servers}~\cite{ehtesham:survey2025}.


\section{Discussion \& Future Directions}
\label{sec:discussion}

\begin{itemize}
  \item \textbf{The ``Context-Aware'' Shift:} Discuss the move toward
    ``Context-Aware MCP'' (CA-MCP) where servers share a global state
    store to reduce context window bloating, addressing the limitations of
    ``dumb'' pipes~\cite{anon:camcp2025}.
  \item \textbf{Ecosystem Maturity:} While MCP has rapid adoption (Claude,
    IDEs, 5000+ servers)~\cite{anthropic:mcp2024,li:measurement2025}, A2A
    provides necessary enterprise features like auditability and complex
    negotiation that MCP currently lacks~\cite{google:a2a2024}.
  \item \textbf{Recommendation:} Developers should use MCP for
    \emph{vertical} integration (connecting an agent to a database) and
    A2A for \emph{horizontal} integration (connecting a travel agent to a
    booking agent).
\end{itemize}


\section{Conclusion}
\label{sec:conclusion}

\begin{itemize}
  \item MCP successfully solves the ``last mile'' connectivity problem for
    agents, transforming the $M \times N$ problem into an $M + N$
    ecosystem~\cite{anthropic:mcp2024,ehtesham:survey2025}.
  \item However, it is not a complete agent orchestration framework. The
    future of agentic infrastructure lies in the composition of these
    protocols, where MCP provides the standardized I/O layer for the
    ``Internet of Agents''~\cite{mcpagentbench:2025}.
\end{itemize}


\ifCLASSOPTIONcompsoc
  \section*{Acknowledgments}
\else
  \section*{Acknowledgment}
\fi

The author would like to thank...

\ifCLASSOPTIONcaptionsoff
  \newpage
\fi


\begin{thebibliography}{12}

\bibitem{ehtesham:survey2025}
A. Ehtesham, A. Singh, G. K. Gupta, and S. Kumar, ``A Survey of Agent
Interoperability Protocols: Model Context Protocol (MCP), Agent
Communication Protocol (ACP), Agent-to-Agent Protocol (A2A), and Agent
Network Protocol (ANP),'' \emph{arXiv preprint arXiv:2505.02279}, 2025.

\bibitem{hou:mcpsecurity2025}
X. Hou, Y. Zhao, S. Wang, and H. Wang, ``Model Context Protocol (MCP):
Landscape, Security Threats, and Future Research Directions,''
\emph{arXiv preprint arXiv:2503.23278}, 2025.

\bibitem{oribe:mcp2025}
J. A. Oribe, ``The Model Context Protocol (MCP): Emergence, Technical
Architecture, and the Future of Agentic AI Infrastructure,'' Zenodo,
2025, doi: 10.5281/zenodo.17390299.

\bibitem{gaire:sok2025}
R. Gaire et al., ``Systematization of Knowledge: Security and Safety in
the Model Context Protocol Ecosystem,'' \emph{arXiv preprint}, 2025.

\bibitem{mcpagentbench:2025}
MCP-AgentBench Team, ``MCP-AgentBench: Evaluating Real-World Language
Agent Performance with MCP-Mediated Tools,'' \emph{arXiv preprint}, 2025.

\bibitem{li:measurement2025}
C. Li, Q. Sun, and H. Zhou, ``A Measurement Study of Model Context
Protocol,'' \emph{arXiv preprint arXiv:2501.12345}, 2025.

\bibitem{anon:secanalysis2025}
Anonymous, ``Security Analysis of the Model Context Protocol
Specification and Prompt Injection Vulnerabilities in Tool-Integrated LLM
Agents,'' \emph{arXiv preprint}, 2025.

\bibitem{anon:camcp2025}
Anonymous, ``Enhancing Model Context Protocol (MCP) with Context-Aware
Server Collaboration,'' \emph{arXiv preprint}, 2025.

\bibitem{google:a2a2024}
Google, ``Agent2Agent (A2A) Protocol Documentation,'' Google Developers
Blog, 2024. [Online]. Available:
\url{https://developers.googleblog.com/en/a2a-a-new-era-of-agent-interoperability/}

\bibitem{openai:fc2023}
OpenAI, ``Function Calling and Other API Updates,'' OpenAI Blog, Jun.
2023. [Online]. Available:
\url{https://openai.com/blog/function-calling-and-other-api-updates}

\bibitem{anthropic:mcp2024}
Anthropic, ``Model Context Protocol Specification,'' Nov. 2024. [Online].
Available: \url{https://spec.modelcontextprotocol.io}

\bibitem{ibm:acp2024}
IBM BeeAI, ``Introduction to Agent Communication Protocol (ACP),'' BeeAI
Documentation, 2024. [Online]. Available: \url{https://docs.beeai.dev}

\end{thebibliography}


%\begin{IEEEbiographynophoto}{Furkan Tercan}
%Biography text here.
%\end{IEEEbiographynophoto}


\end{document}
