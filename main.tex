%% main.tex
%% IEEE Computer Society Journal Paper
%% Based on bare_jrnl_compsoc.tex V1.4b by Michael Shell

\documentclass[10pt,journal,compsoc]{IEEEtran}

\ifCLASSOPTIONcompsoc
  \usepackage[nocompress]{cite}
\else
  \usepackage{cite}
\fi

\usepackage{amsmath}
\usepackage{amssymb}
\usepackage{url}

\hyphenation{op-tical net-works semi-conduc-tor}


\begin{document}

\title{Beyond the M$\times$N Problem: A Comparative Architectural Analysis of\\
Model Context Protocol (MCP) and Agent-to-Agent (A2A) Interaction Patterns}

\author{Ömer Furkan Tercan%
%\IEEEcompsocitemizethanks{%
%  \IEEEcompsocthanksitem F. Tercan is an independent researcher.\protect\\
%  E-mail: furkan@tercan.dev
%}%
%\thanks{Manuscript received \today.}
}

% E-mail: omer.tercan@helsinki.fi

%\markboth{Journal of \LaTeX\ Class Files,~Vol.~1, No.~1, 2026}%
%{Tercan: Beyond the M$\times$N Problem: MCP and A2A Interaction Patterns}

\IEEEtitleabstractindextext{%
\begin{abstract}
The transition from conversational AI to agentic systems has created an
``$M \times N$ integration problem,'' where $M$ models require bespoke
connectors for $N$ data sources, leading to fragmented
infrastructure~\cite{oribe:mcp2025,ayyagari:mcpagentic2025}.
While recent surveys have cataloged emerging standards like MCP,
Agent-to-Agent (A2A), and Agent Communication Protocol
(ACP)~\cite{ehtesham:survey2025}, there remains a lack of rigorous
comparison regarding their architectural trade-offs. This paper analyzes
MCP against its primary contenders---Google's A2A protocol and OpenAI's
REST-based function calling---to determine the specific utility of
stateful versus stateless interaction patterns. We utilize a
``Systematization of Knowledge'' (SoK)
approach~\cite{hou:mcpsecurity2025} combined with a constructive
evaluation of a standardized ``Research Assistant'' task
graph~\cite{ganesh:mcp2025} implemented across three protocols. Analysis
reveals that MCP's persistent, stateful connection model (JSON-RPC) offers
superior context management for local-first and high-fidelity tool
use~\cite{hou:mcpsecurity2025,maloyan:breaking2026}, whereas A2A excels
in decentralized, trust-based task delegation between autonomous
entities~\cite{ehtesham:survey2025,hou:mcpsecurity2025}. We propose a unified ``Agent
Protocol Stack,'' arguing that MCP and A2A are complementary
layers---MCP as the standard for \emph{tool execution} and A2A for
\emph{agent collaboration}---rather than mutually exclusive
competitors~\cite{mitra:stack2026}.
\end{abstract}

\begin{IEEEkeywords}
Model Context Protocol, MCP, Agent-to-Agent, A2A, ACP, agent
interoperability, JSON-RPC, stateful protocols, Systematization of
Knowledge.
\end{IEEEkeywords}}

\maketitle
\IEEEdisplaynontitleabstractindextext
\IEEEpeerreviewmaketitle


\IEEEraisesectionheading{\section{Introduction}\label{sec:introduction}}

\IEEEPARstart{T}{he} rapid proliferation of Large Language Models (LLMs)
has necessitated standardized interfaces for tools and
memory~\cite{hou:mcpsecurity2025}. Current ad-hoc integrations (manual
API wiring, framework-specific wrappers like LangChain) are brittle and
unscalable~\cite{hou:mcpsecurity2025}.

Previous works, such as ``A Survey of Agent Interoperability
Protocols''~\cite{ehtesham:survey2025}, provided a necessary taxonomy of
the landscape. However, developers currently lack a decision framework for
choosing between \emph{tool-centric} protocols (MCP) and
\emph{agent-centric} protocols (A2A) based on architectural constraints
like latency, security, and state
handling~\cite{hou:mcpsecurity2025,maloyan:breaking2026}.

This paper addresses the following research questions:

\begin{itemize}
  \item \textbf{RQ1 (Architecture):} How do MCP's core
    primitives (Tools, Resources, Prompts)~\cite{ehtesham:survey2025}
    diverge from the ``Agent Card'' and capability negotiation models of
    A2A and ACP~\cite{ehtesham:survey2025}?
  \item \textbf{RQ2 (Security \& State):} How does MCP's stateful
    JSON-RPC session model impact security boundaries compared to
    stateless REST-based function
    calling~\cite{maloyan:breaking2026}?
  \item \textbf{RQ3 (Convergence):} Can these protocols coexist? Is
    there evidence for a layered ``TCP/IP moment'' for the internet of
    agents~\cite{mitra:stack2026}?
\end{itemize}


\section{Background \& Related Work}
\label{sec:background}

\begin{itemize}
  \item \textbf{The ``$M \times N$'' Integration Crisis:} Definition of
    the fragmentation problem where $M$ agents need bespoke connectors
    for $N$ tools, stifling
    innovation~\cite{oribe:mcp2025,ayyagari:mcpagentic2025}.
  \item \textbf{Precursors:} Brief coverage of the Language Server
    Protocol (LSP), which inspired MCP's client-host-server
    topology~\cite{reddit:lspmcp,hou:mcpsecurity2025}, and OpenAI
    Function Calling, which established the de facto REST-based
    standard~\cite{ehtesham:survey2025,hou:mcpsecurity2025}.
  \item \textbf{Existing Surveys:} Cite ``A Survey of Agent
    Interoperability Protocols''~\cite{ehtesham:survey2025} as the
    foundational text. This paper distinguishes itself by moving from
    \emph{description} (what exists) to \emph{architectural evaluation}
    (how it behaves under load/attack).
\end{itemize}


\section{Architectural Analysis}
\label{sec:architecture}

\subsection{Taxonomy of Interaction Patterns}

\begin{itemize}
  \item \emph{Direct Tooling (OpenAI/LangChain):} Ephemeral,
    request-response, stateless. The context must be re-injected every
    turn~\cite{hou:mcpsecurity2025,maloyan:breaking2026}.
  \item \emph{Context-First (MCP):} Persistent connections via JSON-RPC
    (over Stdio or SSE). Decouples ``passive context'' (Resources) from
    ``active execution'' (Tools)~\cite{hou:mcpsecurity2025,ehtesham:survey2025}.
  \item \emph{Peer-Delegation (A2A/ACP):} High-level task handoff,
    asynchronous event loops, and trust-based capability negotiation
    between autonomous peers~\cite{ehtesham:survey2025,hou:mcpsecurity2025}.
\end{itemize}

\subsection{Comparison Matrix}

\begin{itemize}
  \item \textbf{Transport:} MCP's local-first focus (Stdio) vs.\ A2A's
    web-first design (HTTP/SSE)~\cite{maloyan:breaking2026}.
  \item \textbf{Discovery:} MCP's \texttt{initialize} handshake and
    capability declaration vs.\ A2A's ``Agent Card'' lookup vs.\ OpenAI's
    static schema definition.
  \item \textbf{State Handling:} MCP's server-driven resource updates
    (subscriptions) vs.\ REST's client-driven
    polling~\cite{hou:mcpsecurity2025}.
\end{itemize}


\section{Security \& Safety Evaluation}
\label{sec:security}

\subsection{Threat Model Divergence}

\begin{itemize}
  \item \textbf{MCP Risks:} ``Cross-Primitive Escalation'' (using a
    read-only Resource to trigger a Tool
    action)~\cite{oribe:mcp2025} and ``Rug Pulls'' (malicious servers
    changing behavior after trust
    establishment)~\cite{hou:mcpsecurity2025}.
  \item \textbf{Injection Vectors:} Analysis of ``Indirect Prompt
    Injection'' where malicious content in an MCP Resource (e.g., a
    GitHub issue) hijacks the agent's control
    flow~\cite{hou:mcpsecurity2025,badri:robustness2025}.
\end{itemize}

\subsection{The Human-in-the-Loop Problem}

MCP's \texttt{sampling} primitive allows servers to request LLM
completion, creating a bidirectional control flow that complicates
permission boundaries compared to unidirectional REST
APIs~\cite{emergentmind:injection}.


\section{Proposed Convergence: The Agent Protocol Stack}
\label{sec:stack}

This section addresses the ``Future Work'' gap by synthesizing the
protocols~\cite{mitra:stack2026}. The layered model (the ``TCP/IP''
analogy):

\begin{itemize}
  \item \textbf{Layer 3 --- Collaboration} (``The Social Layer''):
    \textbf{A2A / ACP.} Agents talking to Agents. Delegating high-level
    goals (``Plan a trip'')~\cite{ehtesham:survey2025,hou:mcpsecurity2025}.
  \item \textbf{Layer 2 --- Context \& Tools} (``The Hands \& Eyes''):
    \textbf{MCP.} Agents talking to Data/Tools. Executing specific atomic
    actions (``Query database,'' ``Read
    file'')~\cite{oribe:mcp2025,mitra:stack2026}.
  \item \textbf{Layer 1 --- Transport:} HTTP / SSE / JSON-RPC.
\end{itemize}

MCP and A2A are complementary. An A2A agent (the high-level planner)
effectively acts as an \emph{MCP Host} to execute specific sub-tasks
using \emph{MCP Servers}~\cite{mitra:stack2026}.


\section{Discussion \& Future Directions}
\label{sec:discussion}

\begin{itemize}
  \item \textbf{The ``Context-Aware'' Shift:} Discuss the move toward
    ``Context-Aware MCP'' (CA-MCP) where servers share a global state
    store to reduce context window bloating, addressing the limitations of
    ``dumb'' pipes~\cite{jayanti:camcp2026}.
  \item \textbf{Ecosystem Maturity:} While MCP has rapid adoption
    (Claude, IDEs, 5000+
    servers)~\cite{oribe:mcp2025}, A2A provides necessary enterprise
    features like auditability and complex negotiation that MCP currently
    lacks~\cite{hou:mcpsecurity2025}.
  \item \textbf{Recommendation:} Developers should use MCP for
    \emph{vertical} integration (connecting an agent to a database) and
    A2A for \emph{horizontal} integration (connecting a travel agent to a
    booking agent)~\cite{mitra:stack2026}.
\end{itemize}


\section{Conclusion}
\label{sec:conclusion}

\begin{itemize}
  \item MCP successfully solves the ``last mile'' connectivity problem for
    agents, transforming the $M \times N$ problem into an $M + N$
    ecosystem~\cite{oribe:mcp2025,ayyagari:mcpagentic2025}.
  \item However, it is not a complete agent orchestration framework. The
    future of agentic infrastructure lies in the composition of these
    protocols, where MCP provides the standardized I/O layer for the
    ``Internet of Agents''~\cite{mitra:stack2026}.
\end{itemize}


\ifCLASSOPTIONcompsoc
  \section*{Acknowledgments}
\else
  \section*{Acknowledgment}
\fi

The author would like to thank...

\ifCLASSOPTIONcaptionsoff
  \newpage
\fi


\begin{thebibliography}{11}

\bibitem{ehtesham:survey2025}
A. Ehtesham, A. Singh, G. K. Gupta, and S. Kumar, ``A Survey of Agent
Interoperability Protocols: Model Context Protocol (MCP), Agent
Communication Protocol (ACP), Agent-to-Agent Protocol (A2A), and Agent
Network Protocol (ANP),'' \emph{arXiv preprint arXiv:2505.02279}, May
2025.

\bibitem{oribe:mcp2025}
J. A. Oribe, ``The Model Context Protocol (MCP): Emergence, Technical
Architecture, and the Future of Agentic AI Infrastructure,'' Zenodo,
2025, doi: 10.5281/zenodo.17390299.

\bibitem{hou:mcpsecurity2025}
X. Hou, Y. Zhao, S. Wang, and H. Wang, ``Model Context Protocol (MCP):
Landscape, Security Threats, and Future Research Directions,''
\emph{arXiv preprint arXiv:2503.23278}, 2025.

\bibitem{mitra:stack2026}
S. Mitra, ``The Agent Protocol Stack: Why MCP + A2A + A2UI Is the TCP/IP
Moment for Agentic AI,'' \emph{subhadipmitra.com}, Jan. 2026. [Online].
Available: \url{https://subhadipmitra.com/blog/2026/agent-protocol-stack/}

\bibitem{ayyagari:mcpagentic2025}
V. Ayyagari, ``Model Context Protocol for Agentic AI: Enabling Contextual
Interoperability Across Systems,'' \emph{International Journal of
Computational and Experimental Science and Engineering}, vol.~11, no.~3,
pp.~6072--6082, 2025, doi: 10.22399/ijcesen.3678.

\bibitem{ganesh:mcp2025}
K. Ganesh, ``Anthropic's Model Context Protocol (MCP) for AI Applications
and Agents,'' Bluetick Consultants Blog, Mar. 2025. [Online]. Available:
\url{https://www.bluetickconsultants.com/implementing-anthropics-model-context-protocol-mcp-for-ai-applications-and-agents/}

\bibitem{maloyan:breaking2026}
N. Maloyan and D. Namiot, ``Breaking the Protocol: Security Analysis of
the Model Context Protocol Specification and Prompt Injection
Vulnerabilities in Tool-Integrated LLM Agents,'' \emph{arXiv preprint
arXiv:2601.17549}, Jan. 2026.

\bibitem{reddit:lspmcp}
Online Discussion, ``LSP vs MCP: The One True Story to Rule Them All,''
r/mcp, Reddit, 2025. [Online]. Available:
\url{https://www.reddit.com/r/mcp/comments/1joqzpz/lsp_vs_mcp_the_one_true_story_to_rule_them_all/}

\bibitem{badri:robustness2025}
P. R. B. Satya, A. Guyyala, V. Putta, and K. T. Areti, ``Robustness of
Automated AI Agents Against Adversarial Context Injection in MCP,''
\emph{International Journal of Computer Applications}, vol.~187, no.~56,
Nov. 2025, doi: 10.5120/ijca2025925957.

\bibitem{emergentmind:injection}
Emergent Mind, ``Prompt-Based Context Injection Mechanism,'' 2025.
[Online]. Available:
\url{https://www.emergentmind.com/topics/prompt-based-context-injection-mechanism}

\bibitem{jayanti:camcp2026}
M. A. Jayanti and X. Y. Han, ``Enhancing Model Context Protocol (MCP)
with Context-Aware Server Collaboration,'' \emph{arXiv preprint
arXiv:2601.11595}, Jan. 2026.

\end{thebibliography}


%\begin{IEEEbiographynophoto}{Furkan Tercan}
%Biography text here.
%\end{IEEEbiographynophoto}


\end{document}
